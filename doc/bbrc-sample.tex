%\documentclass[a4paper,10pt]{article}
\documentclass{article}

\usepackage{a4wide}
\usepackage[latin1]{inputenc}
\usepackage{fancyhdr}

% % % Watermark
\usepackage{eso-pic}
\usepackage{type1cm}

% % % Figures
% \usepackage{listings}
\usepackage{multirow}	% for tables
\usepackage{graphicx}   % for including EPS
\usepackage{rotating}
\usepackage{subfigure}
\usepackage{url}

% % % Special mathematical fonts
\usepackage{amssymb}
\usepackage{amsmath}
\usepackage{amsfonts}
\usepackage{amsthm}
\usepackage{bm}
\usepackage{srctex}
\usepackage{algorithm}
\usepackage{algpseudocode}
\usepackage{setspace}
\usepackage[compact]{titlesec}
\usepackage{mdwlist}
%\usepackage[left=2cm,top=1cm,right=2cm,nohead,nofoot]{geometry}

\algnotext{EndFor}
\algnotext{EndIf}
\makeindex
\makeatletter

% --- FORMAT ---------------------------------------------------------

% % % Page Style
% Lamport, L., LaTeX : A Documentation Preparation System User's Guide and Reference Manual, Addison-Wesley Pub Co., 2nd edition, August 1994.
\topmargin -2.0cm        % s. Lamport p.163
\oddsidemargin -0.5cm   % s. Lamport p.163
\evensidemargin -0.5cm  % wie oddsidemargin aber fr left-hand pages
\textwidth 17.5cm
\textheight 22.94cm 
\parskip 7.2pt           % spacing zwischen paragraphen
% \renewcommand{\baselinestretch}{2}\normalsize
\parindent 0pt		 % Einrcken von Paragraphen
\headheight 14pt
\pagestyle{fancy}
\lhead{}
\chead{\bfseries}
\rhead{\thepage}
\lfoot{}
\cfoot{}
\rfoot{}
\renewcommand{\textfloatsep}{1.5em}

% % % Proofs: QED-Box
\renewenvironment{proof}[1][\proofname]{\par
  \pushQED{\qed}%
  \normalfont \topsep6\p@\@plus6\p@\relax
  \trivlist
  \item[\hskip\labelsep
        \itshape
    #1\@addpunct{:}]\ignorespaces
}{
  \popQED\endtrivlist\@endpefalse
}
\makeatother

% % % Alphabetic footnote marks
\renewcommand{\thefootnote}{\alph{footnote}}

% % % Watermark
% \makeatletter
% \AddToShipoutPicture{%
% \setlength{\@tempdimb}{.5\paperwidth}%
% \setlength{\@tempdimc}{.5\paperheight}%
% \setlength{\unitlength}{1pt}%
% \makebox(960,1470){%
% \rotatebox{315}{%
% \textcolor[gray]{0.75}{%
% \fontsize{36pt}{72pt}\selectfont{\emph{D R A F T}}}}}
% }
% \makeatother



% --- START DOCUMENT -------------------------------------------------

\begin{document}
\setstretch{1.1}
%\setlength{\parskip}{0pt}
%\setlength{\parsep}{0pt}
%\setlength{\headsep}{0pt}
%\setlength{\topskip}{0pt}
%\setlength{\topmargin}{0pt}
%\setlength{\topsep}{0pt}
%\setlength{\partopsep}{0pt}

%\linespread{0.75}
%
%\titlespacing{\section}{0pt}{*1.5}{*1.5}
%\titlespacing{\subsection}{0pt}{*1.0}{*1.0}
%\titlespacing{\subsubsection}{0pt}{*0.5}{*0.5}

\begin{center}
\begin{huge}Out-Of-Bag Discriminative Graph Mining\end{huge}

Andreas Maunz \\Institute for Physics, Hermann-Herder-Str. 3, 79104 Freiburg, Germany
\end{center}

\section{Abstract}

In class-labeled graph databases, each graph is associated with one from a
finite set of target classes.  This induces associations between subgraphs
occurring in these graphs, and the target classes. The subgraphs with strong
target class associations are called discriminative subgraphs.  In this work,
discriminative subgraphs are repeatedly mined on bootstrap samples of a graph
database in order to estimate the subgraph associations precisely.  This is
done by recording the subgraph frequencies (support values) per class over the
out-of-bag instances of the bootstrap process.  We investigate two different
methods for the approximation of the true underlying support values from these
empirical values, involving sample mean and maximum likelihood estimation.  We
show that both methods significantly improve the process, compared to single
runs of discriminative graph mining, by applying the different methods to
publicly available, chemical datasets.  In computational models of toxicology,
subgraphs (fragments of chemical structure) are routinely used to describe
toxicological properties of molecules. Apart from the subgraph associations being
statistically validated, the subgraph sets created by the proposed methods are
also small, compared to ordinary graph mining, and may thus be beneficial
for statistical models, as well as for inspection by toxicological experts.

\section{Introduction}
It is a fundamental fact that no finite sample can be an exact representation
of the unknown underlying distribution from which the sample was drawn.  For
example, in discriminative graph mining on class-labeled graph databases,
there is no way of inferring the exact correlations between graph fragments
(subgraphs) and the effect symbolized by the target classes, which could serve
to identify the set of patterns that primarily lead to the occurrence of the
effect, e.g. the relations between molecular fragments and exhibited toxicity
of molecules. Finding such relations is, however, a major goal in the research
of chemical reactivity of compounds (toxicology), for example in computational
models, such as structure activity relationships (QSARs) (citation needed).

Moreover, discriminative graph mining is an unstable process, where slight
changes in the sample lead to very different subgraph sets. Fortunately, this
can be rectified by deliberately, repeatedly disturbing the sampling process
and combining estimations over many samples. For example, in bootstrap
sampling, estimates of generalization error obtained from the out-of-bag
instances were shown to drastically improve on estimates obtained from the
sample as a whole \cite{bylander02estimating, breiman96oob}.  

This work employs out-of-bag instances with the aim to more precisely
estimating the frequencies (support values) of the subgraphs on the target
classes that govern the training data than is possible with ordinary
discriminative graph mining. Infrequently occurring subgraphs are filtered
out after the bootstrapping.  For each remaining subgraph, a significance test
is run using the estimated support values to filter out insignificant
subgraphs.  This results in a statistically validated, highly significant,
discriminative subgraph set.  In contrast to bagging, statistical models of the
target classes are not the subject of this work.  However, class-correlated
graph mining is supervised process that should be seen in close context to
model building (see section \ref{s:relatedWork}). For example, a statistical
model could be directly learned on the subgraph descriptors, incorporating
their estimated support values. Moreover, human experts should be able to draw
conclusions from the greatly reduced set. Being subgraphs, they can be easily
matched on the chemical compound structure.

The remainder of this work is structured as follows: We present related work
with a focus on discriminative graph mining (section \ref{s:relatedWork}),
our proposed methods for estimating support values and $p$-values (section
\ref{s:Methods}), as well as algorithmic implementation
(section \ref{s:Algorithm}). Experiments include validation of support
values and $p$-values on publicly available databases of
various sizes and numbers of target classes (section \ref{s:Experiments}),
before we draw conclusions (section \ref{s:Conclusion}).
The contributions of this work are summarized as follows:
\begin{itemize*}
  \item Repeated discriminative graph mining on bootstrap samples of a
    class-labeled graph database is proposed as a means to stabilize estimates
    for subgraph properties, such as support values per class, compared to
    single runs of discriminative graph mining. An important area of
    application is the identification of chemical fragments related to chemical
    reactivity from toxicology.
  \item Two methods for the estimation of support values are described, where both
    handle multiple classes. Significance tests are applied to these
    estimated values and insignificant subgraphs are removed from the results.  
  \item The estimated support values, predicted $p$-values, and predicted class
    associations are empirically validated on molecular databases of various
    sizes.  The results indicate significant improvements over 
    ordinary discriminative graph mining, where the effect is larger the
    smaller the datasets are.  We conclude that out-of-bag estimation has the
    potential to generate concise, highly discriminative descriptor sets for
    use in statistical models and/or expert inspection of the molecular
    fragments.
\end{itemize*}


\section{Related Work}
\label{s:relatedWork}

Out-of-bag methods have been used to robustly estimate node probabilities and
node error rates in decision trees \cite{breiman96oob} as well as the
generalization error and accuracy of bagged predictors
\cite{bylander02estimating}. In the work by Bylander
\cite{bylander02estimating}, generalization error was well modeled by
out-of-bag estimation, and its already small bias could be further reduced by a
correction method, where, in order to correct the prediction of a given
instance, similar out-of-bag instances with the same target class were
employed. However, the method of out-of-bag estimation is not confined to these
examples, and may be used to estimate general statistical properties.

Subgraph mining is often employed as a preprocessing step to statistical
learning, because subgraphs may be useful as descriptors
\cite{bringmann10lego,schietgat09mcsfeatures}. To this end, there is a variety
of well-known statistical learning algorithms available ``off the shelf''
\cite{hall09weka,r08language}. This makes a workflow attractive, where
regularities and patterns are extracted from the data, represented in a unified
format and fed into a machine learning algorithm \cite{hkr03molfea}.  However,
graph mining often produces huge descriptor sets and experts would not be
able to draw conclusions from the vast amount of very similar subgraphs, since
the few useful patterns would be lost in the flood.  Similarly, the
high-dimensional pattern space would prevent machine learning methods from
obtaining models in acceptable time \cite{Hasan_origami:mining}.  Thus,
post-processing would be required to lower redundancy and eliminate the vast
majority of subgraphs \cite{schietgat09mcsfeatures, CMNK01Frequent, Jun04Spin}. 

Subgraph boosting \cite{saigo09gboost} is an integrated method that employs
subsampling internally, alternating between graph mining and model building.
The method presented here is clearly different from subgraph boosting (and from
boosting in general), because it calculates descriptors that can be used in
wide variety of models afterwards. It is similar in that it outputs a small
collection of most discriminative subgraphs. However, the collection computed
here proves to be much more stable, and robust against perturbations of the
dataset.



\section{Methods}
\label{s:Methods}

\subsection{Basic Graph Theory}
\label{ss:BasicGraphTheory}
A graph database is a tuple $(G, \Sigma, a)$, where $G$ is a set of graphs,
$\Sigma \ne \emptyset$  is a totally ordered set of labels and $a: G
\rightarrow C$, $C \subset \mathbb{N}$, is a function that assigns one from a
finite set of class values to every graph in the database.  The set of target
classes $C$ consists of at least two values.  We consider labeled, undirected
graphs, i.e. tuples $g=(V,E,\Sigma,\lambda)$, where $V\ne \emptyset$ is a
finite set of nodes and $E \subseteq V = \{\{v_1, v_2\} \in \{V \times V\}, v_1
\ne v_2\}$ is a set of edges and $\lambda: V\cup E \rightarrow \Sigma$ is a
label function.  We only consider connected graphs here, i.e.  there is a path
between each two nodes in the graph.

A subgraph $g'=(V',E',\Sigma',\lambda')$ of $g$ is a graph such that $V'
\subseteq V$ and $E' \subseteq E$ with $V' \ne \emptyset$ and $E' \ne
\emptyset$, $\lambda'(v_1)=\lambda(v_2)$ whenever $v_1=v_2$, and
$\lambda'(e_1)=\lambda(e_2)$ whenever $e_1=e_2$, for all nodes and edges in
$g'$.  If $g'$ is a subgraph of $g$, we also say that $g'$ covers $g$.  The
subset of the database instances $G$ that $g'$ covers is referred to as the
occurrences of $g'$, and its size as support of $g'$.  As a special case, the
size of the subset of occurrences with $a(g)=i$, for any $g$ in the
occurrences, is referred to as the support of $g'$ for class $i$. Thus, any
subgraph has associated support values per class, ranging each between 0 and
the support of $g'$.

The subgraphs considered in this work are free subtrees. Here, we define a tree
as a graph with exactly $n-1$ edges that connects its $n$ nodes. A free tree is
a tree without a designated root node. For an introduction to tree mining, see
the overview by Chi \emph{et al.} \cite{CMNK01Frequent}.

\subsection{Significance Test}
\label{ss:significance-test}
% general chisq test
For a given subgraph $g$, we seek a $|I| \times 2$ contingency table that lists the
support values per class in the first column and the overall distribution of target
classes in the second column, as in Table \ref{t-ContingencyTableIndTest}.
\begin{table}[t]
  \centering
  \begin{tabular}{|l|l|l|}
    \hline
    ~           &	$g$       & $all$       \\\hline
    class 1	    &	$k_1$     & $|G^1|$     \\\hline
    class 2 	  &	$k_2$     & $|G^2|$     \\\hline
    $\ldots$ 	  &	$\ldots$  & $\ldots$    \\\hline
    class $|I|$	&	$k_{|I|}$ & $|G^{|I|}|$ \\\hline
    $\Sigma$	  &	$k$       & $|G|$       \\\hline
  \end{tabular}
  \caption[]{Contingency table for subgraph $g$.}
  \label{t-ContingencyTableIndTest}
\end{table}
This data allows to check whether $g$'s support values differ
significantly from the overall class distribution.  The $\chi^2_d$ function for
distribution testing, defined as
\begin{equation}
  %\chi^2_d(x,y) = \frac{(y-\frac{xm}{n})^2}{\frac{xm}{n}} + \frac{(x-y-\frac{x(n-m)}{n})^2}{\frac{x(n-m)}{n}},
  \chi^2_d(x,y) = \sum_{i \in \{1,\ldots,|I|\}} \frac{(k_i-E(k_i))^2}{E(k_i)},
  \label{eq:chid}
\end{equation} 
where $E(k_i)=\frac{G^{i}k}{|G|}$ is the expected value of $k_i$, calculates
the sum of squares of deviations from the expected support for all target
classes. The function values are then compared against the $\chi^2$
distribution function to obtain $p$-values and conduct a significance test with
$|I|-1$ degrees of freedom.  The next sections discuss methods to obtain the
$k_i$ entries in the table from the recorded support values (section
\ref{ss:oob-dgm}).  The $|G^{i}|$ values are constants and take the values of
the overall $|G^{i}|$.  This is possible due to the stratified bootstrapping,
which maintains the overall class proportions in each sample (see section
\ref{ss:oob-dgm}). Additionally, $bias(g) := arg max_i(\frac{k_i}{k}/\frac{G^i}{G})$, is determined as the dominant class for $g$.



\subsection{Formulation of Out-Of-Bag Discriminative Graph Mining}
\label{ss:oob-dgm}
We refer to the subset that contains all the graphs $g \in G$
with $a(g)=i$ as $G^i$.  The procedure first splits the graph database randomly
into equal-sized training and test databases $G_{Train}$ and $G_{Test}$.
Subsequently, stratified bootstrapping is performed on the training data, such
that each sample comprises $\vert G_{Train}^i\vert$ graphs associated with
class $i$, drawn with replacement and uniform probability inside each
class $i$.  On average, about 37\% (1/$e$) of training instances (molecules)
will not be drawn in any bootstrap sample (out-of-bag instances). Subgraph mining mines
the subgraphs on the drawn instances, but looks up the support values per
class on the out-of-bag instances, by performing subgraph isomorphism tests
(so-called ``matching'').

After the inital split, the bootstrapping is repeated $N$ times, where in each
iteration, pairs of subgraphs and support values per class
$(g,k_1,\ldots,k_{\vert I\vert})$ are produced, meaning that subgraph $g$
occurs in $k_i$ out-of-bag graphs associated with class $i$. The results are
recorded over the $N$ bootstrap samples, such that for each $g$, the list of
support values is a tuple $(\mathbf{k_1}\ldots\mathbf{k_{\vert I\vert}})$,
where $\mathbf{k_i}$ is a vector $(k_i^1\ldots k_i^N)^T$, containing all the
support values.  

The total support is determined from the class specific support values by
summing up vectors $\mathbf{k_i}$ across target classes:
$\mathbf{k}=\sum_{i=1}^{\vert I\vert} \mathbf{k_i}$. The vectors $\mathbf{k_i}$
are generally sparse, due to the instability of discriminative graph mining,
i.e. perturbations to the dataset (such as bootstrap sampling) yield almost
always a different (but overlapping) selection of subgraphs. To cope with the
variety of rare subgraphs, a fixed threshold removes all subgraphs with less
than $\lfloor0.3*N\rfloor$ entries in the list of support values, after the end
of bootstrapping.

The estimation of support values is performed seperately for each remaining
subgraph. Two methods, described in the next sections, are employed, based either
on the sample mean support per class, and using data of the current subgraph
only, or on a maximum likelihood estimate, involving some of the other
subgraphs. Then, the significance test from section \ref{ss:significance-test}
is run on the estimated support values. 

\subsubsection{Sample Mean Method}
\label{ss:simple-mean}
We set the value of $k_i$ in Table \ref{t-ContingencyTableIndTest} to
$\overline{\mathbf{k_i}}$ for all $i \in \{1,\ldots,|I|\}$, the sample mean
across the entries of vector $\mathbf{k_i}$ (ignoring missing values). The
value of $k$ is the sum of the $k_i$.

\subsubsection{Maximum Likelihood Estimation Method}
\label{ss:MLE}
Here, we employ some of the other subgraphs to form estimates for the $k_i$.
The first step in this process is to extract the subgraphs with the same class
bias as $g$ (see section \ref{ss:significance-test}).  Local ties are broken in
favor of the dominant global class. In case of a further tie on the global
level, one of the globally dominant classes is chosen with uniform probability.
In a second step, the subgraphs with the same class bias as $g$ are used to correct
$g$'s local frequencies by weighting. This approach has some similarity to the
work by Bylander \cite{bylander02estimating}, however, his aim is to correct
instance predictions, and his correction employs similar out-of-bag instances,
whereas our correction happens across bootstraps, and on the subgraphs (not
instances) obtained collectively from all the bootstrap samples.  For each
class, we model the event that each $k_i^j \in \mathbf{k_i}$ would occur for
each of the subgraphs with the same class bias as $g$ as a multinomial
selection process.  More specifically, we determine the class probabilities for
each subgraph $g'$ with the same class bias as $g$ with a maximum likelihood
estimator. It is the smoothed vector of relative class specific support values,
defined as:
\begin{equation}
  \mathbf{\alpha_{g'}} = \left(\frac{1+\vert\mathbf{k_1}\vert_1}{\vert I\vert+\vert\mathbf{k}\vert}_1,\ldots,\frac{1+\vert\mathbf{k_{\vert I\vert}}\vert_1}{\vert I\vert+\vert\mathbf{k}\vert_1}\right)
  \label{eqn:mlexpr}
\end{equation}
where the $\mathbf{k_i}$ and $\mathbf{k}$ pertain to $g'$, and $\vert\cdot\vert_1$ is the one-norm (the sum of the vector elements). Following that, for
each tuple $(k_1^j,\ldots,k_{\vert I\vert}^j)$ pertaining to $g$, a probability distribution is
determined from this collection of multinomials:
\begin{equation}
  p((k_1^j,\ldots,k_{\vert I\vert}^j))=\frac{\sum_{g'} p((k_1^j,\ldots,k_{\vert I\vert}^j); \mathbf{\alpha_{g'}})}{\sum_{g'}1}
  \label{eqn:avgpr}
\end{equation}
Finally, the $k_i^j$ values pertaining to $g$ are corrected in a weighted average
based on this probability distribution:
\begin{equation}
  \overline{\mathbf{k_i}}=\frac{\sum_j k_i^j p((k_1^j,\ldots,k_{\vert I\vert}^j))}{\sum_j p((k_1^j,\ldots,k_{\vert I\vert}^j))}
  \label{eqn:avgki}
\end{equation}
Again, we set the value of $k_i$ in Table \ref{t-ContingencyTableIndTest} to $\overline{\mathbf{k_i}}$.

\subsection{Algorithm}
\label{s:Algorithm}
According to section \ref{ss:oob-dgm}, graph mining proceeds in two steps:
mining the bootstrap sample and -- for each subgraph found -- looking up
support values per class on the out-of-bag instances.  The mining step is
implemented using a discriminative graph mining algorithm of choice. In this
work, an in-house development, called backbone refinement class mining (BBRC)
\cite{maunz09largescale}, is used.  It has high compression potential, which
has been shown theoretically and empirically, while retaining good database
coverage \cite{maunz11efficient}.  It is characterized by low running times in
practice and small result set sizes. In the output, two isomorphic subgraphs
are always represented by the same string identifier.  We employ SMARTS as
identifier, a kind of regular expression to encode molecular fragments as
strings.  This approach allows to store results in a hash structure using SMARTS as
keys.

\renewcommand{\algorithmicrequire}{\textbf{Input:}}
\renewcommand{\algorithmicensure}{\textbf{Output:}}
\begin{algorithm}[t]
  \caption{Estimate subgraph significance on out-of-bag instances}
  \label{alg:bbrc-sample}
  {\small \begin{algorithmic}[1]
  \Require $dataBase, numBoots, minSamplingSupport, method, minFrequencyPerSample, alpha$
  \If{$numBoots=1$}
    \State $[subgraphs, values] \gets BBRC(dataBase, minFrequencyPerSample, alpha)$
  \Else
    \State $hash \gets \{\}$
    \For{$i:=1 \to numBoots$} \Comment{Parallel Processing}
      \State $sample,OOB \gets drawBsSample(dataBase)$
      \State $[subgraphs, values] \gets BBRC(sample, minFrequencyPerSample, alpha)$
      \State $insert(hash,match(subgraphs,OOB))$
    \EndFor
    \State $[subgraphs, values] \gets []$
    \For{$subgraph \in keys(hash)$}
      \If{$length(hash[subgraph]) \geq minSamplingSupport$}
        \State $[candidateSupportValues, candidateBias] \gets method(hash[subgraph])$
        \If{$candidatePValue \gets SignificanceTest(candidateSupportVals) < alpha$}
          \State $subgraphs \gets subgraphs \cup subgraph$
          \State $values \gets values \cup [candidateSupportValues, candidatePValue, candidateBias]$
        \EndIf
      \EndIf
    \EndFor
  \EndIf
  \Ensure $[subgraphs, values]$
\end{algorithmic}}
\end{algorithm}
The algorithm using $numBoots=N$ bootstrap samples is shown in Algorithm
\ref{alg:bbrc-sample}.  We consider the case where $numBoots>1$ first.  Line 4
creates an initially empty hash table to gather results from BBRC mining in
line 7. The resulting subgraphs are matched on the out-of-bag instances in line
8, and the results stored in the hash. On termination of the loop, each hash
entry has at most $N$ support values per class.  Post-processing the results is
very fast and incurs negligible overhead compared to the graph mining step. It
consists of removing subgraphs that (line 11) were not generated often enough
by the sampling process (have not enough entries in the hash table, as
determined by $minSamplingSupport$), or which (line 13) do not significantly
deviate from the overall distribution of classes, as assessed by the $\chi^2$
test (section \ref{ss:significance-test}), with contingency table calculated
according to mean (section \ref{ss:simple-mean}) or maximum likelihood
estimation (section \ref{ss:MLE}) method. If $numBoots=1$, support values per
class are obtained directly from a single BBRC run, without any matching (see
section \ref{ss:Error-estimation}). Note that BBRC and matching compute support
values, $p$-values and biases directly from their respective results.

\section{Experiments}
\label{s:Experiments}

\subsection{Experimental Setup} 
\label{ss:Error-estimation} 
Three methods were compared by their ability to estimate the discriminative
potential of subgraphs, by assessing the deviations between the class specific
support values, $p$-values, and biases of subgraphs, as a) estimated by the respective
method, and b) obtained by matching the subgraphs onto an independent test set.
The methods compared are
\begin{enumerate*} 
  \item Out-of-bag estimation with Algorithm \ref{alg:bbrc-sample}, according
    to section \ref{ss:MLE}. Denote this method by MLE.
  \item Out-of-bag estimation with Algorithm \ref{alg:bbrc-sample}, according
    to section \ref{ss:simple-mean}. Denote this method by MEAN.
  \item Single runs of the BBRC algorithm. Denote this method by BBRC.
\end{enumerate*}

The process was repeated 100 times for methods 1, 2, and 3, with 100 bootstrap samples for methods 1 
and 2. Five different error measures to compare the methods were assessed:
\begin{enumerate*}
  \item $E_1$, the mean of     $ p^B_i -p^T_i \,$                                                                                    over subgraphs, i.e. the bias of $p$-value errors.
  \item $E_2$, the mean of     $ \Big|\,p^B_i -p^T_i \,\Big|$                                                                        over subgraphs, i.e. the absolute $p$-values errors.
  \item $E_3$, the mean of     $ \Big(\,\frac{1}{|I|} \sum_{i=1}^{|I|} \,\Big|\,\frac{k^B_i}{k^B} - \frac{k^T_i}{k^T} \,\Big|\,\Big)$ over subgraphs, i.e. the relative support value errors.
  \item $E_4$, 1 - the mean of $ \delta(bias^B_i, bias^T_ix)$                                                                        over subgraphs, i.e. the fraction with wrongly recognized class bias.
  \item $E_5$, 1 - the mean of $ \delta(p^B_i \le \alpha, p^T_i \le \alpha)$                                                         over subgraphs, i.e. the fraction wrongly recognized as significant.
\end{enumerate*}

The whole procedure is described in Algorithm \ref{alg:pValEstimate}.
\begin{algorithm}[t]
  \caption{Error Measures}
  \label{alg:pValEstimate}
  {\small \begin{algorithmic}[1]
  \Require $graphDatabase, method$ \Comment{method is MLE, MEAN, or BBRC}
  \State $\mathbf{E_1}=\mathbf{E_2}=\mathbf{E_3}=\mathbf{E_4}=\mathbf{E_5}=\left[ \right]$
  \For{$i:=1 \to 100$}
    \State $[trainSet, testSet] \gets splitStratified(graphDatabase,0.5)$ \Comment{Split 50:50}
    \State $numBoots \gets 100;\;\textbf{if}\; method=BBRC\; numBoots \gets 1$
    \State $\left[ subgraphs, values^B \right] \gets Algorithm\,\ref{alg:bbrc-sample}(trainSet,numBoots,alpha=0.05)$ \Comment{$numBoots=1$ for BBRC}
    \State $\left[ subgraphs, values^T \right] \gets match(subgraphs, testSet)$ 
    \For{$j:=1 \to 5$}
      \State $ \mathbf{E_j} \gets \left[ \mathbf{E_j}, errorJ(values^T, values^B) \right]$
    \EndFor
  \EndFor
  \Ensure $\mathbf{E_1},\mathbf{E_2},\mathbf{E_3},\mathbf{E_4},\mathbf{E_5}$
\end{algorithmic}}
\end{algorithm}

Line 1 initializes empty vectors that capture the residuals in estimation.
Inside the main loop, a stratified split (i.e. proportions of target classes
inside each split equal overall proportions) generates a training and a test
set of equal size. The training set is treated by the selected method, which
returns a vector of subgraphs and a vector $\mathbf{values^B}$ of values,
including $p$, class support, and bias (line 5). The subgraphs are matched on
the test set, yielding analogous values $\mathbf{values^T}$ (line 6). Finally, the
residual vectors capture the differences between $\mathbf{values^B}$ and
$\mathbf{values^T}$ by the error measures E1 - E5.

Six molecular, class labeled datasets were used in the experiments. The first
five were drawn from the carcinogenic potency database
(CPDB)\footnote{\url{http://potency.berkeley.edu/cpdb.html}}, namely ``Combined
Carcinogenicity and Mutagenicity'' (MUL, 4 classes, 677 compounds),
``\emph{Salmonella} Mutagenicity'' (SAL, 2 classes, 800 compounds), ``Mouse
Carcinogenicity'' (MOU, 2 classes, 914 compounds), ``Multi-Cell Call'' (MCC, 2
classes, 1050 compounds), and ``Rat Carcinogenicity'' (RAT, 2 classes, 1128
compounds). The sixth is the Kazius/Bursi mutagenicity dataset
\cite{kazius05derivation} (KAZ, 2 classes, 4069 compounds).  The MUL dataset
labels consist of all four cross-combinations of the binary MCC and SAL labels,
for all chemicals with both values for both labels present.

\subsection{Results}
\label{ss:Results}

Table \ref{t:anal} details the results (mean values across bootstraps)
% latex table generated in R 2.14.1 by xtable 1.7-0 package
% Wed Aug 29 08:02:16 2012
\begin{table*}[t]
\begin{center}
  \begin{tabular}{lllllllr}
  \hline
 Assay & Method & E1 & E2 & E3 & E4 & E5 & Subgraphs \\ 
  \hline
  INT & MLE & -0.0107 (0.0610) & 0.0414 (0.0487) & 0.1086 (0.0365) & 0.0029 (0.0137) & 0.1626 (0.1948) & 15.96 \\ 
  INT & MEAN & -0.0062 (0.0553) & 0.0341 (0.0469) & 0.1112 (0.0340) & 0.0013 (0.0085) & 0.1729 (0.1879) & 18.36 \\ 
  INT & BBRC & 0.0630 (0.0955) & 0.0635 (0.0952) & 0.1177 (0.0247) & 0.1733 (0.1071) & 0.4672 (0.1318) & 103.53 \\ 
  MUL & MLE & -0.0129 (0.0154) & 0.0188 (0.0132) & 0.0610 (0.0200) & 0.1457 (0.1723) & 0.0806 (0.1405) & 9.67 \\ 
  MUL & MEAN & -0.0098 (0.0101) & 0.0155 (0.0104) & 0.0694 (0.0239) & 0.1576 (0.1655) & 0.1040 (0.1392) & 10.93 \\ 
  MUL & BBRC & 0.0795 (0.0510) & 0.0795 (0.0510) & 0.1041 (0.0187) & 0.4076 (0.1083) & 0.5616 (0.1214) & 94.95 \\ 
  MOU & MLE & 0.0086 (0.0704) & 0.0397 (0.0607) & 0.0944 (0.0508) & 0.0155 (0.0851) & 0.2357 (0.2719) & 4.51 \\ 
  MOU & MEAN & 0.0161 (0.0593) & 0.0353 (0.0526) & 0.1050 (0.0504) & 0.0132 (0.0716) & 0.2976 (0.2761) & 5.52 \\ 
  MOU & BBRC & 0.1285 (0.1249) & 0.1315 (0.1222) & 0.1209 (0.0355) & 0.1298 (0.1059) & 0.5919 (0.1471) & 48.17 \\ 
  MCC & MLE & 0.0040 (0.0386) & 0.0159 (0.0382) & 0.0664 (0.0348) & 0.0075 (0.0424) & 0.1618 (0.1716) & 12.71 \\ 
  MCC & MEAN & 0.0062 (0.0337) & 0.0194 (0.0327) & 0.0791 (0.0373) & 0.0097 (0.0389) & 0.2089 (0.1743) & 14.75 \\ 
  MCC & BBRC & 0.0636 (0.0511) & 0.0663 (0.0485) & 0.0971 (0.0225) & 0.0583 (0.0524) & 0.5327 (0.1137) & 79.47 \\ 
  RAT & MLE & -0.0028 (0.0125) & 0.0084 (0.0116) & 0.0573 (0.0286) & 0.0022 (0.0141) & 0.0936 (0.1435) & 13.00 \\ 
  RAT & MEAN & -0.0017 (0.0115) & 0.0092 (0.0109) & 0.0659 (0.0317) & 0.0101 (0.0365) & 0.1280 (0.1439) & 14.68 \\ 
  RAT & BBRC & 0.0573 (0.0544) & 0.0621 (0.0507) & 0.0912 (0.0246) & 0.0712 (0.0687) & 0.5059 (0.1253) & 70.01 \\ 
  KAZ & MLE & -0.0000 (0.0000) & 0.0000 (0.0000) & 0.0181 (0.0067) & 0.0000 (0.0000) & 0.0068 (0.0155) & 26.09 \\ 
  KAZ & MEAN & -0.0000 (0.0000) & 0.0000 (0.0000) & 0.0182 (0.0069) & 0.0000 (0.0000) & 0.0052 (0.0140) & 25.57 \\ 
  KAZ & BBRC & 0.0000 (0.0000) & 0.0000 (0.0000) & 0.0181 (0.0066) & 0.0015 (0.0075) & 0.0106 (0.0217) & 25.24 \\ 
   \hline
 \end{tabular}
\caption{Bias and Accuracy}
\label{t:anal}
\end{center}
\end{table*}

%Table \ref{t:sign} details the results ($n$=100). 
%% latex table generated in R 2.14.1 by xtable 1.7-0 package
% Tue Aug 28 11:31:44 2012
\begin{table}[t]
\begin{center}
\begin{tabular}{rllllll}
  \hline
 & INT & MCC & RAT & MUL & KAZ & MOU \\ 
  \hline
MLE vs. BBRC E1 & tw & tw & tw & tw & tw & tw \\ 
  MEAN vs. BBRC E1 & tw & tw & tw & tw & tw & tw \\ 
  MLE vs. MEAN E1 & tw & w & tw & tw &  & tw \\ 
  MLE vs. BBRC E2 &  & tw & tw & tw & tw & tw \\ 
  MEAN vs. BBRC E2 & tw & tw & tw & tw & tw & tw \\ 
  MLE vs. MEAN E2 & tw & w &  & tw & tw & w \\ 
  MLE vs. BBRC E3 &  & tw & tw & tw &  & tw \\ 
  MEAN vs. BBRC E3 &  & tw & tw & tw &  & tw \\ 
  MLE vs. MEAN E3 &  & tw & tw & tw &  & tw \\ 
  MLE vs. BBRC E4 & tw & tw & tw & tw &  & tw \\ 
  MEAN vs. BBRC E4 & tw & tw & tw & tw &  & tw \\ 
  MLE vs. MEAN E4 &  &  & tw &  &  &  \\ 
  MLE vs. BBRC E5 & tw & tw & tw & tw & tw & tw \\ 
  MEAN vs. BBRC E5 & tw & tw & tw & tw & tw & tw \\ 
  MLE vs. MEAN E5 &  & tw & tw & tw &  & tw \\ 
   \hline
\end{tabular}
\caption{Significance ($p$=0.025)}
\label{t:sign}
\end{center}
\end{table}

\begin{figure}[t]
  \begin{tabular}{cc}
   \includegraphics[width=10cm]{bp1.eps} & \includegraphics[width=5.5cm]{lp1.eps} \\
   \includegraphics[width=10cm]{bp2.eps} & \includegraphics[width=5.5cm]{lp2.eps} \\
   \includegraphics[width=10cm]{bp3.eps} & \includegraphics[width=5.5cm]{lp3.eps} \\
  \end{tabular}
  \caption{Results}
  \label{fig:bplp13}
\end{figure}
\begin{figure}[t]
  \begin{tabular}{cc}
   \includegraphics[width=10cm]{bp4.eps} & \includegraphics[width=5.5cm]{lp4.eps} \\
   \includegraphics[width=10cm]{bp5.eps} & \includegraphics[width=5.5cm]{lp5.eps} \\
  \end{tabular}
  \caption{Results}
  \label{fig:bplp45}
\end{figure}
Figure \ref{fig:bplp13} and Figure \ref{fig:bplp45} plot the results. 

We first consider E1, E2, and E3. 
For four of the six datasets (apart from SAL and KAZ), there is a clear improvement when out-of-bag estimation is applied (MLE or MEAN), compared to single runs of discriminative subgraph mining (BBRC).
In all these cases, the differences are statistically significant at the $\alpha=0.025$ level. 
This means that here, the proposed out-of-bag estimation of support values improves significantly on mining subgraphs directly (BBRC), regardless of the specific method (MEAN or MLE). 
However, E3 is not significantly different for SAL and KAZ between BBRC and MEAN or MLE. Instead, since SAL's E1 and E2 are significantly lower for BBRC, compared to MLE and MEAN, BBRC works better for this dataset.
For KAZ, E1 and E2 are practically zero, and E3 is practically the same for all three methods.

Comparing the sampling methods, for the four ``well-behaved'' datasets, there is a draw concerning $p$-value bias (E1), which is, however, low (in absolute values $<0.02$). Concerning absolute $p$-value error (E2), MOU has a quite high value between 0.035 - 0.04 for both MLE and MEAN, which is only 3 - 4 times lower than that of BBRC. The others remain low, with no clear winner. 

However, for E5, the error on significance estimation, MLE performs significantly better than MEAN in all four cases. Interestingly, for E5, MLE and MEAN are both significantly better than BBRC. This means that, despite getting closer to the test values of support and $p$, it still judges a lot of subgraphs as significant at the 0.05 significance level, that are actually insignificant, or vice versa.
E4, the estimation of class bias, is the easiest exercise. However, MLE and MEAN perform significantly better than BBRC for all datasets except KAZ. The effect is most drastic for the multi-class dataset MUL, but also remarkable for MOU and RAT.

Table \ref{t:anal} also gives the mean number of subgraphs generated in the last column. There is a general trend for the smaller numbers of subgraphs generated with the sampling methods with shrinking dataset sizes. This reflects the higher uncertainty due to the lack of data. However, as the line plots in Figures \ref{fig:bplp13} and \ref{fig:bplp45} show, there is a trend towards a higher gap in errors between the sampling methods on the one hand, and BBRC on the other hand with shrinking dataset sizes. It becomes also clear from these plots, that SAL behaves different than the other datasets (as discussed before).


\section{Conclusion}
\label{s:Conclusion}


\bibliography{bbrc-sample}
\bibliographystyle{plain}

\end{document} 
