\documentclass[a4paper,10pt]{article}

\usepackage{a4wide}
\usepackage[latin1]{inputenc}
\usepackage{fancyhdr}

% % % Watermark
\usepackage{eso-pic}
\usepackage{type1cm}

% % % Figures
% \usepackage{listings}
\usepackage{multirow}	% for tables
\usepackage{graphicx}   % for including EPS
\usepackage{rotating}
% \usepackage{subfigure}

% % % Special mathematical fonts
\usepackage{amssymb}
\usepackage{amsmath}
\usepackage{amsfonts}
\usepackage{amsthm}
\usepackage{bm}
\usepackage{srctex}

\makeindex
\makeatletter

% --- FORMAT ---------------------------------------------------------

% % % Page Style
% Lamport, L., LaTeX : A Documentation Preparation System User's Guide and Reference Manual, Addison-Wesley Pub Co., 2nd edition, August 1994.
\topmargin -2.0cm        % s. Lamport p.163
\oddsidemargin -0.5cm   % s. Lamport p.163
\evensidemargin -0.5cm  % wie oddsidemargin aber fr left-hand pages
\textwidth 17.5cm
\textheight 22.94cm 
\parskip 7.2pt           % spacing zwischen paragraphen
% \renewcommand{\baselinestretch}{2}\normalsize
\parindent 0pt		 % Einrcken von Paragraphen
\headheight 14pt
\pagestyle{fancy}
\lhead{}
\chead{\bfseries}
\rhead{\thepage}
\lfoot{}
\cfoot{}
\rfoot{}
\renewcommand{\textfloatsep}{1.5em}

% % % Proofs: QED-Box
\renewenvironment{proof}[1][\proofname]{\par
  \pushQED{\qed}%
  \normalfont \topsep6\p@\@plus6\p@\relax
  \trivlist
  \item[\hskip\labelsep
        \itshape
    #1\@addpunct{:}]\ignorespaces
}{
  \popQED\endtrivlist\@endpefalse
}
\makeatother

% % % Alphabetic footnote marks
\renewcommand{\thefootnote}{\alph{footnote}}

% % % Watermark
% \makeatletter
% \AddToShipoutPicture{%
% \setlength{\@tempdimb}{.5\paperwidth}%
% \setlength{\@tempdimc}{.5\paperheight}%
% \setlength{\unitlength}{1pt}%
% \makebox(960,1470){%
% \rotatebox{315}{%
% \textcolor[gray]{0.75}{%
% \fontsize{36pt}{72pt}\selectfont{\emph{D R A F T}}}}}
% }
% \makeatother



% --- START DOCUMENT -------------------------------------------------

\begin{document}

\begin{center}
\begin{huge}BBRC Sampling\end{huge}

Andreas Maunz \\Institute for Physics, Hermann-Herder-Str. 3, 79104 Freiburg, Germany
\end{center}

\begin{small}
Sampling BBRC descriptors with non-parametric, stratified bootstrapping seems feasible due to the inherent efficiency of BBRC mining.
Each bootstrap sample of the dataset is processed via a webservice and parallel processing. 
The server is loaded with as many jobs in parallel as there are CPU cores on the client.
A probabilistic chi-square test is performed on each sampled pattern using a poisson MLE estimate.
\end{small}

\section{Basics}
Let $G$ a set of graphs. A graph database is a function $g: G \rightarrow H$, $H \in \mathbb{N}$, $h_i=\vert\{G_j \in G \; \vert\; g(G_i)=i\}\vert$, $\forall i \in H$.

Let a pattern generating process $F: (G,g) \rightarrow (X,k)$, $k=\{k_1,\ldots,k_{\vert H\vert}\}$. $X$ are called patterns (subgraphs), $k_i: X \rightarrow \mathbb{N}_0$ the pattern support function on class $i$.

Apply non-parametric bootstrapping, i.e. draw $n$ samples with replacement from $G$. Ensure that each sample comprises exactly $h_i$ graphs $G$ with $g(G)=i$, with uniform probability $1/h_i$ inside each class $i \in H$ (stratification).

The result of bootstrapping is a pattern set X, and sets of functions $k^{(1)},\ldots,k^{(n)}$.

\section{Frequencies}
Define
\begin{itemize}
  \item the frequency of support value $j$ on class $i$ as 
    \begin{equation*}
        w_{i,j}:=\sum_{l=1}^n \delta_{k_i^{(l)}(x),\, j}\, , \; \forall i \in H, j \in \mathbb{N}_0
    \end{equation*}
  \item the frequency of support value $j$ on class $i$, given the sum of $j$'s equals $h$, as 
    \begin{equation*}
      w_{i,j,h}:=\sum_{l=1}^n \delta_{k_i^{(l)}(x),\, j} * \delta_{\sum_{i=1}^{\vert H \vert}k_i^{(l)}(x),\, h}\, , \; \forall i \in H, \; j,h \in \mathbb{N}_0
    \end{equation*}
  \item the cumulative frequency of support value $j$ over classes as 
    \begin{equation*}
      w_j(x):=\sum_{i \in H}w_{i,j}(x)
    \end{equation*}
\end{itemize}

\section{Probability Distributions}
Below treats a single pattern only. (drop dependency on $x$ for better readability).

Consider the finite set of support values $J \subset \mathbb{N}_0$ with $w_j > 0, \; j \in J$, and the categorical distribution for support value $j \in J$ with parameter $W_J:=\sum_{j \in J} w_j$:
\begin{align*}
  p(j|W_J) &= w_j/W_J
\end{align*}

Consider the \textsc{Poisson} distribution for support value $k$ in class $i$, given the sum of support values equals $j$.
\begin{align*}
  p(k\vert\lambda_{i,j}) &= \operatorname{Pois}\left(\lambda_{i,j}\right), \text{where}\\
  \lambda_{i,j}&=\sum_{l \in \mathbb{N}} l*w_{i,l,j} / \sum_{l \in \mathbb{N}} w_{i,l,k}
\end{align*}
Being the sample mean, parameter $\lambda_{i,j}$ is the MLE of the Poisson distribution.

\section{Significance Testing}
Consider the $\chi^2$ distribution test, defined as

\begin{align*}
  \chi^2 = \sum_{i \in H}\; \left( \sum_{j \in J}p(j|W_j) \int_0^{\infty}dk\; p(k|\lambda_{i,j}) \frac{(k-E_j(k_i)^2}{E_j(k_i)} \right)
\end{align*}

where 

\begin{align*}
  E_j(k_i) = \frac{j * h_i}{\vert G\vert}
\end{align*}

the expected support value on class $i$ when the sum of support values is $j$. Then, $p(\chi^2)=\operatorname{Chi-Square}\left(\vert H-1\vert\right)$


\end{document} 
