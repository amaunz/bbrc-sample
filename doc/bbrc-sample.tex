% This is "sig-alternate.tex" V1.9 April 2009
% This file should be compiled with V2.4 of "sig-alternate.cls" April 2009
%
% This example file demonstrates the use of the 'sig-alternate.cls'
% V2.4 LaTeX2e document class file. It is for those submitting
% articles to ACM Conference Proceedings WHO DO NOT WISH TO
% STRICTLY ADHERE TO THE SIGS (PUBS-BOARD-ENDORSED) STYLE.
% The 'sig-alternate.cls' file will produce a similar-looking,
% albeit, 'tighter' paper resulting in, invariably, fewer pages.
%
% ----------------------------------------------------------------------------------------------------------------
% This .tex file (and associated .cls V2.4) produces:
%       1) The Permission Statement
%       2) The Conference (location) Info information
%       3) The Copyright Line with ACM data
%       4) NO page numbers
%
% as against the acm_proc_article-sp.cls file which
% DOES NOT produce 1) thru' 3) above.
%
% Using 'sig-alternate.cls' you have control, however, from within
% the source .tex file, over both the CopyrightYear
% (defaulted to 200X) and the ACM Copyright Data
% (defaulted to X-XXXXX-XX-X/XX/XX).
% e.g.
% \CopyrightYear{2007} will cause 2007 to appear in the copyright line.
% \crdata{0-12345-67-8/90/12} will cause 0-12345-67-8/90/12 to appear in the copyright line.
%
% ---------------------------------------------------------------------------------------------------------------
% This .tex source is an example which *does* use
% the .bib file (from which the .bbl file % is produced).
% REMEMBER HOWEVER: After having produced the .bbl file,
% and prior to final submission, you *NEED* to 'insert'
% your .bbl file into your source .tex file so as to provide
% ONE 'self-contained' source file.
%
% ================= IF YOU HAVE QUESTIONS =======================
% Questions regarding the SIGS styles, SIGS policies and
% procedures, Conferences etc. should be sent to
% Adrienne Griscti (griscti@acm.org)
%
% Technical questions _only_ to
% Gerald Murray (murray@hq.acm.org)
% ===============================================================
%
% For tracking purposes - this is V1.9 - April 2009

\documentclass{sig-alternate}
  \pdfpagewidth=8.5truein
  \pdfpageheight=11truein

\makeatletter
\newif\if@restonecol
\makeatother
\let\algorithm\relax
\let\endalgorithm\relax


\usepackage{multirow}	% for tables
\usepackage{graphicx}   % for including EPS
\usepackage{rotating}
\usepackage{subfigure}
\usepackage{url}

% % % Special mathematical fonts
\usepackage{amssymb}
\usepackage{amsmath}
\usepackage{amsfonts}
\usepackage{bm}
\usepackage{srctex}
\usepackage[linesnumbered,algo2e,noend]{algorithm2e}
\usepackage{algpseudocode}
\usepackage{setspace}
%\usepackage[compact]{titlesec}
\usepackage{mdwlist}
%\usepackage[left=2cm,top=1cm,right=2cm,nohead,nofoot]{geometry}

\algnotext{EndFor}
\algnotext{EndIf}

\begin{document}
%
% --- Author Metadata here ---
\conferenceinfo{SAC'13}{March 18-22, 2013, Coimbra, Portugal.}
\CopyrightYear{2013} % Allows default copyright year (2002) to be over-ridden - IF NEED BE.
\crdata{978-1-4503-1656-9/13/03}  % Allows default copyright data (X-XXXXX-XX-X/XX/XX) to be over-ridden.
% --- End of Author Metadata ---

\title{Out-Of-Bag Discriminative Graph Mining}
%
% You need the command \numberofauthors to handle the 'placement
% and alignment' of the authors beneath the title.
%
% For aesthetic reasons, we recommend 'three authors at a time'
% i.e. three 'name/affiliation blocks' be placed beneath the title.
%
% NOTE: You are NOT restricted in how many 'rows' of
% "name/affiliations" may appear. We just ask that you restrict
% the number of 'columns' to three.
%
% Because of the available 'opening page real-estate'
% we ask you to refrain from putting more than six authors
% (two rows with three columns) beneath the article title.
% More than six makes the first-page appear very cluttered indeed.
%
% Use the \alignauthor commands to handle the names
% and affiliations for an 'aesthetic maximum' of six authors.
% Add names, affiliations, addresses for
% the seventh etc. author(s) as the argument for the
% \additionalauthors command.
% These 'additional authors' will be output/set for you
% without further effort on your part as the last section in
% the body of your article BEFORE References or any Appendices.

\numberofauthors{3} %  in this sample file, there are a *total*
% of EIGHT authors. SIX appear on the 'first-page' (for formatting
% reasons) and the remaining two appear in the \additionalauthors section.
%
\author{
% You can go ahead and credit any number of authors here,
% e.g. one 'row of three' or two rows (consisting of one row of three
% and a second row of one, two or three).
%
% The command \alignauthor (no curly braces needed) should
% precede each author name, affiliation/snail-mail address and
% e-mail address. Additionally, tag each line of
% affiliation/address with \affaddr, and tag the
% e-mail address with \email.
%
% 1st. author
\alignauthor
Andreas Maunz\\
       \affaddr{Institute for Physics}\\
       \affaddr{Hermann-Herder-Str. 3}\\
       \affaddr{79194 Freiburg, Germany}\\
       \email{andreas@maunz.de}
% 2nd. author
\alignauthor David Vorgrimmler\\
       \affaddr{in-silico Toxicology}\\
       \affaddr{Altkircherstr. 4}\\
       \affaddr{4054 Basel, Switzerland}\\
       \email{vorgrimmler@in-silico.ch}
% 3rd. author
\alignauthor 
Christoph Helma\\
       \affaddr{in-silico Toxicology}\\
       \affaddr{Altkircherstr. 4}\\
       \affaddr{4054 Basel, Switzerland}\\
       \email{helma@in-silico.ch}
}

\maketitle
\begin{abstract}
  In class-labeled graph databases, each graph is associated with one from a
finite set of target classes.  This induces associations between subgraphs
occurring in these graphs, and the target classes. The subgraphs with strong
target class associations are called discriminative subgraphs.  In this work,
discriminative subgraphs are repeatedly mined on bootstrap samples of a graph
database in order to estimate the subgraph associations precisely.  This is
done by recording the subgraph frequencies (support values) per class over the
out-of-bag instances of the bootstrap process.  We investigate two different
methods for the approximation of the true underlying support values from these
empirical values, involving sample mean and maximum likelihood estimation.  We
show that both methods significantly improve the process, compared to single
runs of discriminative graph mining, by applying the different methods to
publicly available, chemical datasets.  In computational models of toxicology,
subgraphs (fragments of chemical structure) are routinely used to describe
toxicological properties of molecules. Apart from the subgraph associations being
statistically validated, the subgraph sets created by the proposed methods are
also small, compared to ordinary graph mining, and may thus be beneficial
for statistical models, as well as for inspection by toxicological experts.


\end{abstract}



% A category with the (minimum) three required fields
\category{H.2.8}{Database Applications}{Data Mining, Statistical Databases}
\terms{Theory, Experimentation, Performance}

\section{Introduction}
\label{s:Introduction}
Given a class-labeled graph database, discriminative graph mining is a
supervised learning task with the goal to extract graph fragments (subgraphs)
with strong associations to the classes, according to statistical constraints
set by the user. For example, the subgraphs may be molecular fragments that
induce toxicity. Finding such associations is indeed a major goal in toxicology
\cite{kazius05derivation}. However, discriminative graph mining yields large
result sets, even for very tight constraints on subgraph significance ($p$-values). Moreover, it is an unstable
process, i.e. slight changes in the sample may lead to substantially different subgraph
sets. 

A bagged predictor consists of several aggregate predictors, each trained on a
dedicated bootstrap sample of the training data. Bagged predictors have the
potential for considerably higher predictive accuracy, compared to single
predictors, especially in the context of unstable prediction methods
\cite{breiman96oob}. Bagged predictor estimates may be obtained from the out-of-bag
instances, which saves computational effort compared to crossvalidation on the
bootstrap samples. For example, out-of-bag estimation of node errors in
decision trees could improve on estimates obtained from the training data as a
whole \cite{breiman96oob}.

This work extends discriminative subgraph mining by out-of-bag estimation for
subgraph frequencies (support values) on the target classes that govern the
training data, and by aggregating subgraphs in the output that pass the
statistical constraints.  Infrequently occurring subgraphs are filtered out
during the bootstrapping, which removes the instability to a large extent. The
result is a small set of statistically validated, highly significant,
discriminative subgraphs, which may be useful for diagnostic or predictive
modelling, or for expert inspection. 

The remainder of this work is structured as follows: We present related work
with a focus on discriminative graph mining (section \ref{s:relatedWork}),
our proposed methods for out-of-bag estimation of support values and $p$-values (section
\ref{s:Methods}), as well as algorithmic implementation
(section \ref{s:Algorithm}). Experiments include validation of support
values and $p$-values on publicly available databases of
various sizes and numbers of target classes (section \ref{s:Experiments}),
before we draw conclusions (section \ref{s:Conclusion}).
The contributions of this work are summarized as follows:
\begin{itemize*}
  \item Repeated discriminative graph mining on bootstrap samples of a
    class-labeled graph database is proposed as a means to stabilize estimates
    for subgraph properties, such as support values per class, compared to
    single runs of discriminative graph mining. The estimation is performed
    using the out-of-bag instances of the individual bootstrap samples.
  \item Two methods for the estimation of support values are described, where both
    handle multiple classes. Significance tests are applied to these
    estimated values and insignificant subgraphs are removed from the results.  
  \item The estimated support values, predicted $p$-values, and predicted class
    associations are empirically validated on molecular databases of various
    sizes.  The results indicate significant improvements over 
    ordinary discriminative graph mining, where the effect is larger the
    smaller the datasets are.  We conclude that out-of-bag estimation has the
    potential to generate concise, highly discriminative subgraph sets for
    use in statistical models and/or expert inspection.
\end{itemize*}


\section{Related Work}
\label{s:relatedWork}

Out-of-bag methods have been used to robustly estimate node probabilities and
node error rates in decision trees \cite{breiman96oob} as well as the
generalization error and accuracy of bagged predictors. In the work by Bylander
\cite{bylander02estimating}, generalization error was well modeled by
out-of-bag estimation, and its already small bias could be further reduced by a
correction method, where, in order to correct the prediction of a given
instance, similar out-of-bag instances with the same target class were
employed. However, the method of out-of-bag estimation is not confined to these
examples, and may be used to also estimate other statistical properties in
supervised learning.

Discriminative subgraph mining is often employed as a preprocessing step to
statistical learning, because such subgraphs may be useful as descriptors
\cite{bringmann10lego}. Nowadays, there is a variety of well-known statistical
learning algorithms available ``off the shelf'', which makes a workflow
attractive where subgraphs are extracted from the data, represented in a
unified format, and fed into a machine learning algorithm \cite{hkr03molfea}.
Usually, discriminative subgraphs are mined using a user-defined significance
threshold. A subgraph is in the result, if it passes a statistical test with
regard to its association to the target classes.  However, graph mining often
produces huge descriptor sets even with very tight bounds on discriminative
power, which would prevent machine learning methods from obtaining models in
acceptable time \cite{Hasan_origami:mining}.  Thus, post-processing would be
required to lower redundancy and eliminate the vast majority of subgraphs
\cite{Jun04Spin}. 

Subgraph boosting \cite{saigo09gboost} is an integrated method that employs
subsampling internally, alternating between graph mining and model building.
The method presented here is clearly different from boosting, because it
calculates descriptors that can be used in wide variety of models afterwards.
It is similar in that it calculates a small collection of most discriminative
subgraphs. However, it is also stable against perturbations of the
dataset, which is not the case in boosting.


\section{Methods}
\label{s:Methods}

\subsection{Basic Graph Theory}
\label{ss:BasicGraphTheory}
A graph database is a tuple $(G, \Sigma, a)$, where $G$ is a set of graphs,
$\Sigma \ne \emptyset$  is a totally ordered set of labels and $a: G
\rightarrow C$, $C \subset \mathbb{N}$, is a function that assigns one from a
finite set of class values to every graph in the database.  The set of target
classes $C$ consists of at least two values.  We consider labeled, undirected
graphs, i.e. tuples $g=(V,E,\Sigma,\lambda)$, where $V\ne \emptyset$ is a
finite set of nodes and $E \subseteq V = \{\{v_1, v_2\} \in \{V \times V\}, v_1
\ne v_2\}$ is a set of edges and $\lambda: V\cup E \rightarrow \Sigma$ is a
label function.  We only consider connected graphs here, i.e.  there is a path
between each two nodes in the graph.

A graph $g'=(V',E',\Sigma',\lambda')$ subgraph-isomorphic to $g$ if $V'
\subseteq V$ and $E' \subseteq E$ with $V' \ne \emptyset$ and $E' \ne
\emptyset$, $\lambda'(v_1)=\lambda(v_2)$ whenever $v_1=v_2$, and
$\lambda'(e_1)=\lambda(e_2)$ whenever $e_1=e_2$, for all nodes and edges in
$g'$. Then, graph $g'$ is also referred to as a subgraph of $g$ -- we also say
that $g'$ covers $g$.  The subset of the database instances $G$ that $g'$
covers is referred to as the occurrences of $g'$, and its size as support of
$g'$.  As a special case, the size of the subset of occurrences with $a(g)=i$,
for any $g$ in the occurrences, is referred to as the support of $g'$ for class
$i$. Thus, any subgraph has associated support values per class, ranging each
between 0 and the support of $g'$.

The subgraphs considered in this work are free subtrees. Here, we define a tree
as a graph with exactly $n-1$ edges that connects its $n$ nodes. A free tree is
a tree without a designated root node. For an introduction to tree mining, see
the overview by Chi \emph{et al.} \cite{CMNK01Frequent}.

\subsection{Significance Test}
\label{ss:significance-test}
% general chisq test
For a given subgraph $g$, we seek a $|I| \times 2$ contingency table that lists the
support values per class in the first column and the overall distribution of target
classes in the second column, as in Table \ref{t-ContingencyTableIndTest}.
\begin{table}[t]
  \centering
  \begin{tabular}{|l|l|l|}
    \hline
    ~           &	$g$       & $all$       \\\hline
    class 1	    &	$k_1$     & $|G^1|$     \\\hline
    class 2 	  &	$k_2$     & $|G^2|$     \\\hline
    $\ldots$ 	  &	$\ldots$  & $\ldots$    \\\hline
    class $|I|$	&	$k_{|I|}$ & $|G^{|I|}|$ \\\hline
    $\Sigma$	  &	$k$       & $|G|$       \\\hline
  \end{tabular}
  \caption[]{Contingency table for subgraph $g$.}
  \label{t-ContingencyTableIndTest}
\end{table}
This data allows to check whether $g$'s support values differ
significantly from the overall class distribution.  The $\chi^2_d$ function for
distribution testing, defined as
\begin{equation}
  %\chi^2_d(x,y) = \frac{(y-\frac{xm}{n})^2}{\frac{xm}{n}} + \frac{(x-y-\frac{x(n-m)}{n})^2}{\frac{x(n-m)}{n}},
  \chi^2_d(x,y) = \sum_{i \in \{1,\ldots,|I|\}} \frac{(k_i-E(k_i))^2}{E(k_i)},
  \label{eq:chid}
\end{equation} 
where $E(k_i)=\frac{G^{i}k}{|G|}$ is the expected value of $k_i$, calculates
the sum of squares of deviations from the expected support for all target
classes. The function values are then compared against the $\chi^2$
distribution function to obtain $p$-values and conduct a significance test with
$|I|-1$ degrees of freedom.  The next sections discuss methods to obtain the
$k_i$ entries in the table from the recorded support values (section
\ref{ss:oob-dgm}).  The $|G^{i}|$ values are constants and take the values of
the overall $|G^{i}|$.  This is possible due to the stratified bootstrapping,
which maintains the overall class proportions in each sample (see section
\ref{ss:oob-dgm}). Additionally, $bias(g) := arg max_i(\frac{k_i}{k}/\frac{G^i}{G})$, is determined as the dominant class for $g$.



\subsection{Formulation of Out-Of-Bag Discriminative Graph Mining}
\label{ss:oob-dgm}
We refer to the subset that contains all the graphs $g \in G$
with $a(g)=i$ as $G^i$.  The procedure first splits the graph database randomly
into equal-sized training and test databases $G_{Train}$ and $G_{Test}$.
Subsequently, stratified bootstrapping is performed on the training data, such
that each sample comprises $\vert G_{Train}^i\vert$ graphs associated with
class $i$, drawn with replacement and uniform probability inside each
class $i$.  On average, about 37\% (1/$e$) of training instances (molecules)
will not be drawn in any bootstrap sample (out-of-bag instances). Subgraph mining mines
the subgraphs on the drawn instances, but looks up the support values per
class on the out-of-bag instances, by performing subgraph isomorphism tests
(so-called ``matching'').

After the inital split, the bootstrapping is repeated $N$ times, where in each
iteration, pairs of subgraphs and support values per class
$(g,k_1,\ldots,k_{\vert I\vert})$ are produced, meaning that subgraph $g$
occurs in $k_i$ out-of-bag graphs associated with class $i$. The results are
recorded over the $N$ bootstrap samples, such that for each $g$, the list of
support values is a tuple $(\mathbf{k_1}\ldots\mathbf{k_{\vert I\vert}})$,
where $\mathbf{k_i}$ is a vector $(k_i^1\ldots k_i^N)^T$, containing all the
support values.  

The total support is determined from the class specific support values by
summing up vectors $\mathbf{k_i}$ across target classes:
$\mathbf{k}=\sum_{i=1}^{\vert I\vert} \mathbf{k_i}$. The vectors $\mathbf{k_i}$
are generally sparse, due to the instability of discriminative graph mining,
i.e. perturbations to the dataset (such as bootstrap sampling) yield almost
always a different (but overlapping) selection of subgraphs. To cope with the
variety of rare subgraphs, a fixed threshold removes all subgraphs with less
than $\lfloor0.3*N\rfloor$ entries in the list of support values, after the end
of bootstrapping.

The estimation of support values is performed seperately for each remaining
subgraph. Two methods, described in the next sections, are employed, based either
on the sample mean support per class, and using data of the current subgraph
only, or on a maximum likelihood estimate, involving some of the other
subgraphs. Then, the significance test from section \ref{ss:significance-test}
is run on the estimated support values. 

\subsubsection{Sample Mean Method}
\label{ss:simple-mean}
We set the value of $k_i$ in Table \ref{t-ContingencyTableIndTest} to
$\overline{\mathbf{k_i}}$ for all $i \in \{1,\ldots,|I|\}$, the sample mean
across the entries of vector $\mathbf{k_i}$ (ignoring missing values). The
value of $k$ is the sum of the $k_i$.

\subsubsection{Maximum Likelihood Estimation Method}
\label{ss:MLE}
Here, we employ some of the other subgraphs to form estimates for the $k_i$.
The first step in this process is to extract the subgraphs with the same class
bias as $g$ (see section \ref{ss:significance-test}).  Local ties are broken in
favor of the dominant global class. In case of a further tie on the global
level, one of the globally dominant classes is chosen with uniform probability.
In a second step, the subgraphs with the same class bias as $g$ are used to correct
$g$'s local frequencies by weighting. This approach has some similarity to the
work by Bylander \cite{bylander02estimating}, however, his aim is to correct
instance predictions, and his correction employs similar out-of-bag instances,
whereas our correction happens across bootstraps, and on the subgraphs (not
instances) obtained collectively from all the bootstrap samples.  For each
class, we model the event that each $k_i^j \in \mathbf{k_i}$ would occur for
each of the subgraphs with the same class bias as $g$ as a multinomial
selection process.  More specifically, we determine the class probabilities for
each subgraph $g'$ with the same class bias as $g$ with a maximum likelihood
estimator. It is the smoothed vector of relative class specific support values,
defined as:
\begin{equation}
  \mathbf{\alpha_{g'}} = \left(\frac{1+\vert\mathbf{k_1}\vert_1}{\vert I\vert+\vert\mathbf{k}\vert}_1,\ldots,\frac{1+\vert\mathbf{k_{\vert I\vert}}\vert_1}{\vert I\vert+\vert\mathbf{k}\vert_1}\right)
  \label{eqn:mlexpr}
\end{equation}
where the $\mathbf{k_i}$ and $\mathbf{k}$ pertain to $g'$, and $\vert\cdot\vert_1$ is the one-norm (the sum of the vector elements). Following that, for
each tuple $(k_1^j,\ldots,k_{\vert I\vert}^j)$ pertaining to $g$, a probability distribution is
determined from this collection of multinomials:
\begin{equation}
  p((k_1^j,\ldots,k_{\vert I\vert}^j))=\frac{\sum_{g'} p((k_1^j,\ldots,k_{\vert I\vert}^j); \mathbf{\alpha_{g'}})}{\sum_{g'}1}
  \label{eqn:avgpr}
\end{equation}
Finally, the $k_i^j$ values pertaining to $g$ are corrected in a weighted average
based on this probability distribution:
\begin{equation}
  \overline{\mathbf{k_i}}=\frac{\sum_j k_i^j p((k_1^j,\ldots,k_{\vert I\vert}^j))}{\sum_j p((k_1^j,\ldots,k_{\vert I\vert}^j))}
  \label{eqn:avgki}
\end{equation}
Again, we set the value of $k_i$ in Table \ref{t-ContingencyTableIndTest} to $\overline{\mathbf{k_i}}$.

\subsection{Algorithm}
\label{s:Algorithm}
According to section \ref{ss:oob-dgm}, graph mining proceeds in two steps:
mining the bootstrap sample and -- for each subgraph found -- looking up
support values per class on the out-of-bag instances.  The mining step is
implemented using a discriminative graph mining algorithm of choice. In this
work, our algorithm backbone refinement class mining (BBRC) is used
\cite{maunz09largescale}. It has high compression potential, which has been
shown theoretically and empirically, while retaining good database coverage
\cite{maunz11efficient}. BBRC takes a significance threshold, as well as a
minimum support parameter. In its output, two isomorphic subgraphs are always
represented by the same string identifier.  We employ SMARTS as identifier, a
kind of regular expression to encode molecular fragments as strings.  This
approach allows to store results in a hash structure using SMARTS as keys.
%\renewcommand{\algorithmicrequire}{\textbf{Input:}}
%\renewcommand{\algorithmicensure}{\textbf{Output:}}
\begin{algorithm2e*}[t]
  \KwData{$dataBase, numBoots, minSamplingSupport, method, minFrequencyPerSample, alpha$}
  \KwResult{$[subgraphs, values]$}
  \If{$numBoots=1$}{
     $[subgraphs, values] \leftarrow BBRC(dataBase, minFrequencyPerSample, alpha)$\;
  }
  \Else{
    $hash \leftarrow \{\}$\;
    \For{$i:=1 \to numBoots$ (Parallel Processing)} { 
      $sample,OOB \leftarrow drawBsSample(dataBase)$\;
      $[subgraphs, values] \leftarrow BBRC(sample, minFrequencyPerSample, alpha)$\;
      $insert(hash,match(subgraphs,OOB))$\;
    }
    $[subgraphs, values] \leftarrow []$\;
    \For{$subgraph \in keys(hash)$}{
      \If{$length(hash[subgraph]) \geq minSamplingSupport$}{
         $[candidateSupportValues, candidateBias] \leftarrow method(hash[subgraph])$\;
        \If{$candidatePValue \leftarrow SignificanceTest(candidateSupportVals) < alpha$}{
          $subgraphs \leftarrow subgraphs \cup subgraph$\;
          $values \leftarrow values \cup [candidateSupportValues, candidatePValue, candidateBias]$\;
        }
      }
    }
  }
  \caption{Estimate subgraph significance on out-of-bag instances\label{alg:bbrc-sample}}
\end{algorithm2e*}
The algorithm using $numBoots=N$ bootstrap samples is shown in Algorithm
\ref{alg:bbrc-sample}.  We consider the case where $numBoots>1$ first.  Line 4
creates an initially empty hash table to gather results from BBRC mining in
line 7. The resulting subgraphs are matched on the out-of-bag instances in line
8, and the results stored in the hash. On termination of the loop, each hash
entry has at most $N$ support values per class.  Post-processing the results is
very fast and incurs negligible overhead compared to the graph mining step. It
consists of removing subgraphs that (line 11) were not generated often enough
by the sampling process (have not enough entries in the hash table, as
determined by $minSamplingSupport$), or which (line 13) do not significantly
deviate from the overall distribution of classes, as assessed by the $\chi^2$
test (section \ref{ss:significance-test}), with contingency table calculated
according to mean (section \ref{ss:simple-mean}) or maximum likelihood
estimation (section \ref{ss:MLE}) method. If $numBoots=1$, support values per
class are obtained directly from a single BBRC run, without any matching (see
section \ref{ss:Error-estimation}). Note that BBRC and matching compute support
values, $p$-values and biases directly from their respective results.

\section{Experiments}
\label{s:Experiments}

\subsection{Experimental Setup} 
\label{ss:Error-estimation} 
Three methods were compared by their ability to estimate the discriminative
potential of subgraphs, by assessing the deviations between the class specific
support values, $p$-values, and biases of subgraphs, as a) estimated by the respective
method, and b) obtained by matching the subgraphs onto an independent test set.
The methods compared are
\begin{enumerate*} 
  \item Out-of-bag estimation with Algorithm \ref{alg:bbrc-sample}, according
    to section \ref{ss:MLE}. Denote this method by MLE.
  \item Out-of-bag estimation with Algorithm \ref{alg:bbrc-sample}, according
    to section \ref{ss:simple-mean}. Denote this method by MEAN.
  \item Single runs of the BBRC algorithm. Denote this method by BBRC.
\end{enumerate*}

The process was repeated 100 times for methods 1, 2, and 3, with 100 bootstrap samples for methods 1 
and 2. Five different error measures to compare the methods were assessed:
\begin{enumerate*}
  \item $E_1$, the mean of     $ p^B_i -p^T_i \,$                                                                                    over subgraphs, i.e. the bias of $p$-value errors.
  \item $E_2$, the mean of     $ \Big|\,p^B_i -p^T_i \,\Big|$                                                                        over subgraphs, i.e. the absolute $p$-values errors.
  \item $E_3$, the mean of     $ \Big(\,\frac{1}{|I|} \sum_{i=1}^{|I|} \,\Big|\,\frac{k^B_i}{k^B} - \frac{k^T_i}{k^T} \,\Big|\,\Big)$ over subgraphs, i.e. the relative support value errors.
  \item $E_4$, 1 - the mean of $ \delta(bias^B_i, bias^T_ix)$                                                                        over subgraphs, i.e. the fraction with wrongly recognized class bias.
  \item $E_5$, 1 - the mean of $ \delta(p^B_i \le \alpha, p^T_i \le \alpha)$                                                         over subgraphs, i.e. the fraction wrongly recognized as significant.
\end{enumerate*}
\begin{algorithm2e*}[t]
  \KwData{$graphDatabase, method$ (method is MLE, MEAN, or BBRC)} 
  \KwResult{$\mathbf{E_1},\mathbf{E_2},\mathbf{E_3},\mathbf{E_4},\mathbf{E_5}$}
  $\mathbf{E_1}=\mathbf{E_2}=\mathbf{E_3}=\mathbf{E_4}=\mathbf{E_5}=\left[ \right]$\;
  \For{$i:=1 \to 100$}{
    $[trainSet, testSet] \leftarrow splitStratified(graphDatabase,0.5)$ \;
    $numBoots \leftarrow 100;\;\textbf{if}\; method=BBRC\; numBoots \leftarrow 1$\;
    $\left[ subgraphs, values^B \right] \leftarrow Algorithm\,\ref{alg:bbrc-sample}(trainSet,numBoots,alpha=0.05)$\;
    $\left[ subgraphs, values^T \right] \leftarrow match(subgraphs, testSet)$ \;
    \For{$j:=1 \to 5$}{
      $\mathbf{E_j} \leftarrow \left[ \mathbf{E_j}, errorJ(values^T, values^B) \right]$\;
    }
  }
  \caption{Error Measures\label{alg:pValEstimate}}
\end{algorithm2e*}
The whole procedure is described in Algorithm \ref{alg:pValEstimate}.
Line 1 initializes empty vectors that capture the residuals in estimation.
Inside the main loop, a stratified split (i.e. proportions of target classes
inside each split equal overall proportions) generates a training and a test
set of equal size. The training set is treated by the selected method, which
returns a vector of subgraphs and a vector $\mathbf{values^B}$ of values,
including $p$, class support, and bias (line 5). The subgraphs are matched on
the test set, yielding analogous values $\mathbf{values^T}$ (line 6). Finally, the
residual vectors capture the differences between $\mathbf{values^B}$ and
$\mathbf{values^T}$ by the error measures E1 - E5.

Six molecular, class labeled datasets were used in the experiments. 
Four were drawn from the carcinogenic potency database
(CPDB)\footnote{\url{http://potency.berkeley.edu/cpdb.html}}, namely 
``Combined Carcinogenicity and Mutagenicity'' (MUL, 4 classes, 677 compounds),
``Mouse Carcinogenicity'' (MOU, 2 classes, 914 compounds), 
``Multi-Cell Call'' (MCC, 2 classes, 1050 compounds), and 
``Rat Carcinogenicity'' (RAT, 2 classes, 1128 compounds). 
MUL's labels consist of the four cross-combinations of binary carcinogenicity
and mutagenicity labels from the CPDB, for all chemicals with both values for
both labels present.
A rather small dataset, describing human intestinal absorption (INT, 3 classes, 458 compounds) \cite{Suenderhauf10Combinatorial}, as well as
the rather large Kazius/Bursi mutagenicity dataset (KAZ, 2 classes, 4069 compounds), \cite{kazius05derivation} were also used.

\subsection{Results}
\label{ss:Results}

Table \ref{t:anal} details the results (mean values across bootstraps)
*% latex table generated in R 2.14.1 by xtable 1.7-0 package
% Mon Aug 20 16:04:34 2012
\begin{table}[t]
\begin{center}
\begin{tabular}{rlllllllr}
  \hline
 & Assay & Method & E1 & E2 & E3 & E4 & E5 & Subgraphs \\ 
  \hline
1 & MUL & MLE & -0.0127 ( 0.0155 ) & 0.0187 ( 0.0134 ) & 0.0027 ( 0.0017 ) & 1.0000 ( 0.0000 ) & 0.9193 ( 0.1417 ) & 9.68 \\ 
  2 & MUL & MEAN & -0.0094 ( 0.0101 ) & 0.0153 ( 0.0105 ) & 0.0026 ( 0.0016 ) & 1.0000 ( 0.0000 ) & 0.8957 ( 0.1404 ) & 10.93 \\ 
  3 & MUL & BBRC & 0.0783 ( 0.0509 ) & 0.0783 ( 0.0509 ) & 0.0014 ( 0.0009 ) & 1.0000 ( 0.0000 ) & 0.4399 ( 0.1231 ) & 94.95 \\ 
  4 & SAL & MLE & -0.0174 ( 0.0082 ) & 0.0202 ( 0.0071 ) & 0.0039 ( 0.0031 ) & 1.0000 ( 0.0000 ) & 0.9531 ( 0.0737 ) & 24.85 \\ 
  5 & SAL & MEAN & -0.0145 ( 0.0075 ) & 0.0166 ( 0.0066 ) & 0.0037 ( 0.0031 ) & 1.0000 ( 0.0000 ) & 0.9566 ( 0.0624 ) & 25.82 \\ 
  6 & SAL & BBRC & 0.0034 ( 0.0059 ) & 0.0070 ( 0.0063 ) & 0.0028 ( 0.0023 ) & 1.0000 ( 0.0000 ) & 0.7750 ( 0.1031 ) & 58.61 \\ 
  7 & MOU & MLE & 0.0086 ( 0.0704 ) & 0.0397 ( 0.0607 ) & 0.0044 ( 0.0031 ) & 1.0000 ( 0.0000 ) & 0.7643 ( 0.2719 ) & 4.51 \\ 
  8 & MOU & MEAN & 0.0161 ( 0.0593 ) & 0.0353 ( 0.0526 ) & 0.0040 ( 0.0030 ) & 1.0000 ( 0.0000 ) & 0.7024 ( 0.2761 ) & 5.52 \\ 
  9 & MOU & BBRC & 0.1285 ( 0.1249 ) & 0.1315 ( 0.1222 ) & 0.0028 ( 0.0019 ) & 1.0000 ( 0.0000 ) & 0.4081 ( 0.1471 ) & 48.17 \\ 
  10 & MCC & MLE & 0.0040 ( 0.0388 ) & 0.0160 ( 0.0384 ) & 0.0034 ( 0.0024 ) & 1.0000 ( 0.0000 ) & 0.8389 ( 0.1723 ) & 12.67 \\ 
  11 & MCC & MEAN & 0.0063 ( 0.0339 ) & 0.0194 ( 0.0328 ) & 0.0034 ( 0.0022 ) & 1.0000 ( 0.0000 ) & 0.7922 ( 0.1749 ) & 14.75 \\ 
  12 & MCC & BBRC & 0.0629 ( 0.0509 ) & 0.0657 ( 0.0483 ) & 0.0024 ( 0.0018 ) & 1.0000 ( 0.0000 ) & 0.4680 ( 0.1140 ) & 79.47 \\ 
  13 & RAT & MLE & -0.0028 ( 0.0125 ) & 0.0084 ( 0.0116 ) & 0.0035 ( 0.0025 ) & 1.0000 ( 0.0000 ) & 0.9061 ( 0.1442 ) & 12.97 \\ 
  14 & RAT & MEAN & -0.0017 ( 0.0116 ) & 0.0092 ( 0.0109 ) & 0.0033 ( 0.0023 ) & 1.0000 ( 0.0000 ) & 0.8724 ( 0.1446 ) & 14.68 \\ 
  15 & RAT & BBRC & 0.0574 ( 0.0547 ) & 0.0623 ( 0.0510 ) & 0.0026 ( 0.0020 ) & 1.0000 ( 0.0000 ) & 0.4945 ( 0.1258 ) & 70.01 \\ 
  16 & KAZ & MLE & -0.0000 ( 0.0000 ) & 0.0000 ( 0.0000 ) & 0.0034 ( 0.0025 ) & 1.0000 ( 0.0000 ) & 0.9932 ( 0.0155 ) & 26.09 \\ 
  17 & KAZ & MEAN & -0.0000 ( 0.0000 ) & 0.0000 ( 0.0000 ) & 0.0034 ( 0.0025 ) & 1.0000 ( 0.0000 ) & 0.9948 ( 0.0140 ) & 25.57 \\ 
  18 & KAZ & BBRC & 0.0000 ( 0.0000 ) & 0.0000 ( 0.0000 ) & 0.0032 ( 0.0024 ) & 1.0000 ( 0.0000 ) & 0.9894 ( 0.0217 ) & 25.24 \\ 
   \hline
\end{tabular}
\caption{Bias and Accuracy}
\label{t:anal}
\end{center}
\end{table}

%Table \ref{t:sign} details the results ($n$=100). 
%% latex table generated in R 2.14.1 by xtable 1.7-0 package
% Fri Jul 20 13:50:57 2012
\begin{table}[t]
\begin{center}
\begin{tabular}{rllllllll}
  \hline
 & SAL E1 & SAL E2 & RAT E1 & RAT E2 & MCC E1 & MCC E2 & KAZ E1 & KAZ E2 \\ 
  \hline
MLE vs. BBRC & tw & tw & tw & tw & tw & tw & tw & tw \\ 
  MEAN vs. BBRC & tw & tw & tw & tw & tw & tw & tw & tw \\ 
  MLE vs. MEAN &  &  & tw & t & tw & tw & tw & tw \\ 
   \hline
\end{tabular}
\caption{Significance ($p$=0.001)}
\label{t:sign}
\end{center}
\end{table}

\begin{figure*}[t]
  \begin{tabular}{ccccc}
    \includegraphics[width=5.5cm]{lp1.eps}  &
    \includegraphics[width=5.5cm]{lp2.eps}  &
    \includegraphics[width=5.5cm]{lp3.eps}  &
    \includegraphics[width=5.5cm]{lp4.eps}  &
    \includegraphics[width=5.5cm]{lp5.eps}  \\
  \end{tabular}
  \caption{Results}
  \label{fig:lp}
\end{figure*}
%
%\begin{figure}[t]
%  \begin{tabular}{cc}
%   \includegraphics[width=10cm]{bp1.eps} & \includegraphics[width=5.5cm]{lp1.eps} \\
%   \includegraphics[width=10cm]{bp2.eps} & \includegraphics[width=5.5cm]{lp2.eps} \\
%   \includegraphics[width=10cm]{bp3.eps} & \includegraphics[width=5.5cm]{lp3.eps} \\
%  \end{tabular}
%  \caption{Results}
%  \label{fig:bplp13}
%\end{figure}
%\begin{figure}[t]
%  \begin{tabular}{cc}
%   \includegraphics[width=10cm]{bp4.eps} & \includegraphics[width=5.5cm]{lp4.eps} \\
%   \includegraphics[width=10cm]{bp5.eps} & \includegraphics[width=5.5cm]{lp5.eps} \\
%  \end{tabular}
%  \caption{Results}
%  \label{fig:bplp45}
%\end{figure}
Figure \ref{fig:lp} plots the results. 
%Figure \ref{fig:bplp13} and Figure \ref{fig:bplp45} plot the results. 

We first consider E1, E2, and E3. 
For four of the six datasets (apart from SAL and KAZ), there is a clear improvement when out-of-bag estimation is applied (MLE or MEAN), compared to single runs of discriminative subgraph mining (BBRC).
In all these cases, the differences are statistically significant at the $\alpha=0.025$ level. 
This means that here, the proposed out-of-bag estimation of support values improves significantly on mining subgraphs directly (BBRC), regardless of the specific method (MEAN or MLE). 
However, E3 is not significantly different for SAL and KAZ between BBRC and MEAN or MLE. Instead, since SAL's E1 and E2 are significantly lower for BBRC, compared to MLE and MEAN, BBRC works better for this dataset.
For KAZ, E1 and E2 are practically zero, and E3 is practically the same for all three methods.

Comparing the sampling methods, for the four ``well-behaved'' datasets, there is a draw concerning $p$-value bias (E1), which is, however, low (in absolute values $<0.02$). Concerning absolute $p$-value error (E2), MOU has a quite high value between 0.035 - 0.04 for both MLE and MEAN, which is only 3 - 4 times lower than that of BBRC. The others remain low, with no clear winner. 

However, for E5, the error on significance estimation, MLE performs significantly better than MEAN in all four cases. Interestingly, for E5, MLE and MEAN are both significantly better than BBRC. This means that, despite getting closer to the test values of support and $p$, it still judges a lot of subgraphs as significant at the 0.05 significance level, that are actually insignificant, or vice versa.
E4, the estimation of class bias, is the easiest exercise. However, MLE and MEAN perform significantly better than BBRC for all datasets except KAZ. The effect is most drastic for the multi-class dataset MUL, but also remarkable for MOU and RAT.

Table \ref{t:anal} also gives the mean number of subgraphs generated in the last column. There is a general trend for the smaller numbers of subgraphs generated with the sampling methods with shrinking dataset sizes. This reflects the higher uncertainty due to the lack of data. However, as the line plots in Figure \ref{fig:lp} show, there is a trend towards a higher gap in errors between the sampling methods on the one hand, and BBRC on the other hand with shrinking dataset sizes. It becomes also clear from these plots, that SAL behaves different than the other datasets (as discussed before).


\section{Conclusion}
\label{s:Conclusion}
Especially for small graph databases, which are typical in toxicology, the
proposed methods can significantly improve on the estimation of statistical
quantities, compared to ordinary discriminative graph mining. For larger
datasets, this advantage shrinks, which we attribute to increasing accuracy of
ordinary discriminative graph mining in the presence of more data.


\bibliography{bbrc-sample}
\bibliographystyle{plain}

\end{document}
